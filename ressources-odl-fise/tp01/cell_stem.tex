% ------------------------------------------------------------------------------
Cellules souches
% ------------------------------------------------------------------------------

Une cellule souche embryonnaire est une cellule souche pluripotente issue de la
masse cellulaire interne d’un embryon préimplantatoire au stade de blastocyste.
Un embryon humain atteint le stade de blastocyste 4 à 5 jours après la fécondation
et consiste en un amas de 50 à 150 cellules (masse cellulaire interne et
trophectoderme). L'isolation de la masse cellulaire interne requiert de détruire
le blastocyste.

Les cellules souches embryonnaires sont une source quasi-parfaite pour les greffes
et l'ingénierie tissulaire. La capacité d'une cellule souche à générer l'ensemble
des cellules du corps en fait un outil clé pour la recherche sur les maladies
humaines, notamment génétique, ou pour tester in vitro la toxicologie de médicaments.

Cependant, l'isolation de cellules souches embryonnaires pose un questionnement
éthique qui nécessite notamment d'établir si un embryon au stade pré-implantatoire
possède les mêmes droits légaux et moraux qu'un être humain plus développé. A
l'inverse, ne pas utiliser ces cellules capables de sauver des vies est-il juste,
alors que les embryons pré-implantatoires surnuméraires ne seront pas utilisé et
donc détruits ? Il n'existe pas de consensus et la législation sur l'obtention et
l'utilisation en recherche de ces cellules souches embryonnaires varie selon les pays.

Une des alternatives aux cellules souches embryonnaires est d'utiliser des cellules
souches pluripotentes induites qui sont générées à partir de cellules différenciées
(par exemple, des cellules de peau) et ne présentent donc pas ce même dilemme éthique.

